\documentclass[a4paper]{article}

% Paquetes
\usepackage[utf8]{inputenc}
\usepackage{babel}
\usepackage[margin=2cm, top=2cm, includefoot]{geometry}
\usepackage{graphicx} % Gestión de imágenes
\usepackage[table,xcdraw]{xcolor} % Gestión colores
\usepackage[most]{tcolorbox} % Inserción de cuadros (portada)
\usepackage{fancyhdr} % Definir estilos de pagina
\usepackage[hidelinks]{hyperref} % Gestión hipervinculos
\usepackage{listings} % Inserción de código
\usepackage{parskip} % Arreglo tabulación automática
\usepackage[figurename=Figura]{caption} % Gestión nombre del caption
\usepackage{smartdiagram} % Inserción de diagramas
\usepackage{zed-csp} % Inserción de esquemas
\usepackage{datetime}
%\usepackage{mwe}

% Declaración de variables
\newcommand{\myLogoCover}{\VAR{icon1}}
\newcommand{\myLogoProyect}{\VAR{icon2}}
\newcommand{\myTitle}{Automatic Report}
\newcommand{\mySubtitle}{\href{https://telegram.me/procamora_scan_bot}{@procamora\_scan\_bot}}
\newcommand{\myName}{\href{https://github.com/procamora/bot_scan_networks}{Pablo Rocamora}}
\newcommand{\myMail}{pablojoserocamora@gmail.com}

% Definicion del mes en el idioma deseado
\newcommand{\myMonth}{%
  \ifcase\month   % 0
    \or January   % 1
    \or February  % 2
    \or March     % 3
    \or April     % 4
    \or May       % 5
    \or June      % 6
    \or July      % 7
    \or August    % 8
    \or September % 9
    \or October   % 10
    \or November  % 11
    \or December  % 12
  \fi}
\newcommand{\myDate}{\currenttime\space \number\day\space \myMonth \number\year}


% Definicion metadados PDF
\hypersetup{
    pdfauthor = {\myName~(\myMail)},
    pdftitle = {\myTitle},
}


% Adicionales
\setlength{\headheight}{40.2pt}
\pagestyle{fancy}
\fancyhf{}
\lhead{\includegraphics[height=1.5cm,keepaspectratio]{\myLogoCover}}
\rhead{\includegraphics[height=1.5cm,keepaspectratio]{\myLogoProyect}}
\renewcommand{\headrulewidth}{3pt}

\renewcommand{\headrule}{\hbox to\headwidth{\color{greenPortada}\leaders\hrule height \headrulewidth\hfill}}


\AtBeginDocument{ % Cambiar nombre de los disitntos 
    \renewcommand{\tablename}{Table}
    \renewcommand{\figurename}{Figure}
    \renewcommand{\lstlistingname}{Code} % Cambiar nombre de los codigos
    
     \renewcommand{\contentsname}{Index}
    \renewcommand{\listfigurename}{Index of \figurename s}
    \renewcommand{\listtablename}{Index of \tablename s} 
    \renewcommand{\lstlistlistingname}{Index of \lstlistingname s}
}





% Definición de colores
%\definecolor{greenPortada}{HTML}
\definecolor{greenPortada}{RGB}{105, 168, 79} % hex: {69A84F}
\definecolor{codegreen}{rgb}{0,0.6,0}
\definecolor{codegray}{rgb}{0.5,0.5,0.5}
\definecolor{codepurple}{rgb}{0.58,0,0.82}
\definecolor{backcolor}{rgb}{0.95,0.95,0.92}

% Estilos propios
\lstdefinestyle{mystyle}{
    backgroundcolor=\color{backcolor},
    commentstyle=\color{codegreen},
    keywordstyle=\color{magenta},
    numberstyle=\tiny\color{codegray},
    stringstyle=\color{codepurple},
    basicstyle=\ttfamily\footnotesize,
    breakatwhitespace=false,
    breaklines=true,
    captionpos=b,
    keepspaces=true,
    numbers=left,
    numbersep=5pt,
    showspaces=false,
    showstringspaces=false,
    showtabs=false,
    tabsize=2
}
\lstset{style=mystyle}


% Comienzo del documento
\begin{document}
\cfoot{\thepage}
% Inicio Portada
\begin{titlepage}
	\centering
	%\vspace*{\fill}\includegraphics[height=2cm,keepaspectratio]{\myLogoCover} {\scshape\LARGE \textbf{\myName}\par}\vspace*{\fill}
    
    \begin{tabular}{ll}
    \raisebox{-.4\height}{\includegraphics[height=2cm,keepaspectratio]{\myLogoCover}} & { \hspace{8cm} \scshape\LARGE {\myName}} \\
	\end{tabular}
    
    \par\vspace{1cm}
	\vfill
    
    {\rule{\linewidth}{0.5mm}\vspace{0.5cm}} % Linea negra 1
    
    {\Huge\bfseries\textcolor{greenPortada}{\myTitle}}
	\par\vspace{0.5cm}
	{\scshape\LARGE \textbf{\mySubtitle}}

    {\vspace{0.5cm}\rule{\linewidth}{0.5mm}} % Linea negra 2
    
	\vfill\vfill
	\includegraphics[width=\textwidth,height=8cm,keepaspectratio]{\myLogoProyect}\par\vspace{1cm}
	\vfill\vfill
    
    % Cuadrado con disclaimer
	\begin{tcolorbox}[colback=red!5!white,colframe=red!75!black]
		\centering
		This document is confidential and contains sensitive information.
		\\
		It should not be printed or shared with third parties.
	\end{tcolorbox}
    
	\vfill\vfill
	{\large \myDate \par}
	\vfill
\end{titlepage}
% Fin Portada

	



\clearpage
%\tableofcontents	     % Índice
%\listoffigures		     % Índice de figuras
%\listoftables		     % Índice de tablas
%\lstlistoflistings     % Índice de códigos
\clearpage





\section{Hosts}



\subsection{Hosts online}


\begin{table}[!hbt]
	\centering
	\begin{tabular}{|c|c|c|c|c|}
		\hline
		\textbf{IP} & \textbf{MAC} & \textbf{VENDOR} & \textbf{DESCRIPTION} & \textbf{NETWORK} \\ \hline

		%% for x in hosts
		%% if hosts[x].active
		\VAR{hosts[x].ip} & \VAR{hosts[x].mac} & \VAR{hosts[x].vendor} & \VAR{hosts[x].description} & \VAR{hosts[x].network} \\ \hline
		%% endif
		%% endfor

	\end{tabular}
	\caption{Hosts online}
	\label{tab:my-table}
\end{table}



\subsection{Hosts offline}

\begin{table}[!hbt]
	\centering
	\begin{tabular}{|c|c|c|c|c|}
		\hline
		\textbf{IP} & \textbf{MAC} & \textbf{VENDOR} & \textbf{DESCRIPTION} & \textbf{NETWORK} \\ \hline

		%% for x in hosts
		%% if not hosts[x].active
		\VAR{hosts[x].ip} & \VAR{hosts[x].mac} & \VAR{hosts[x].vendor} & \VAR{hosts[x].description} & \VAR{hosts[x].network} \\ \hline
		%% endif
		%% endfor

	\end{tabular}
	\caption{Hosts offline}
	\label{tab:my-table2}
\end{table}













\clearpage

\section{Troubleshooting}


\begin{center}
\smartdiagram[sequence diagram]{ip address show, ip neigh show, ip route list}
\end{center}



\subsection{Interfaces}


\begin{lstlisting}[language=Bash, caption=ip address show]
\VAR{interfaces}
\end{lstlisting}

\subsection{ARP}


\begin{lstlisting}[language=Bash, caption=ip neigh show]
\VAR{arp}
\end{lstlisting}


\subsection{Routing}


\begin{lstlisting}[language=Bash, caption=ip route list]
\VAR{routes}
\end{lstlisting}





\end{document}




